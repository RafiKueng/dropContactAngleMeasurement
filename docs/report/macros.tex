
\newcommand{\spl}{SpaghettiLens\xspace}

% shorcuts for refs (use capital for beginning of sentence)
% first 3 are for the real layz people..
\newcommand{\fref}[1]{\ref{fig:#1}}
\newcommand{\sref}[1]{\ref{sec:#1}}
\newcommand{\tref}[1]{\ref{tab:#1}}
\newcommand{\lref}[1]{\ref{lst:#1}}
\newcommand{\uref}[1]{\ref{uml:#1}}
\newcommand{\figref}[1]{Figure~\ref{fig:#1}}
\newcommand{\secref}[1]{Section~\ref{sec:#1}}
\newcommand{\tabref}[1]{Table~\ref{tab:#1}}
\newcommand{\lstref}[1]{Listing~\ref{lst:#1}}
\newcommand{\umlref}[1]{UML~Diagram~\ref{uml:#1}}
\newcommand{\seqref}[1]{Figure~\ref{seq:#1}}
\newcommand{\Figref}[1]{Figure~\ref{fig:#1}}
\newcommand{\Secref}[1]{Section~\ref{sec:#1}}
\newcommand{\Tabref}[1]{Table~\ref{tab:#1}}
\newcommand{\Lstref}[1]{Listing~\ref{lst:#1}}
\newcommand{\Umlref}[1]{UML~Diagram~\ref{uml:#1}}
\newcommand{\Seqref}[1]{Figure~\ref{seq:#1}}

% reference to ranges of listings (for multipage..)
% \lstrefr[n_pages]{base filename}
\newcommand{\lstrefr}[2][1]{\crefrange{lst:#2_1}{lst:#2_#1}}
\newcommand{\Lstrefr}[2][1]{\Crefrange{lst:#2_1}{lst:#2_#1}}


\newcommand{\code}[2]{%
\begin{listing}[!ht]%
  \centering%
  \input{code/#1}%
  \caption{#2}%
  \label{lst:#1}%
\end{listing}%
}

% insert multiple pages of code
% \codep[npages]{base file name}{caption}
\newcommand{\codep}[3][1]{%
  \foreach \index in {1, ..., #1} {%
    \code{#2_\index}{#3 (page \index\xspace of #1)}
  }%
}

\newcommand{\uml}[3][width=\textwidth]{%
\begin{figure}[!ht]%
  \centering%
  \includegraphics[#1]{uml/#2.pdf}%
  \caption{#3}%
  \label{uml:#2}%
\end{figure}%
}


\newcommand{\fig}[3][width=\figwidth]{%
\begin{figure}[!ht]%
  \centering%
  \includegraphics[#1]{fig/#2.pdf}%
  \caption{#3}%
  \label{fig:#2}%
\end{figure}%
}

\newcommand{\seq}[3][width=\textwidth]{%
\begin{figure}[!ht]%
  \centering%
  \includegraphics[#1]{seq/#2.pdf}%
  \caption{#3}%
  \label{seq:#2}%
\end{figure}%
}

% small code fragments

% class / object name
\newcommand{\C}[1]{%
\protect\path{#1}%
}
%method
\newcommand{\M}[1]{%
\protect\path{#1}%
}
% function
\newcommand{\F}[1]{%
\protect\path{#1}%
}
% some other code
\newcommand{\T}[1]{%
\protect\path{#1}%
}
% strings
\newcommand{\str}[1]{%
{\ttfamily "#1"}%
}
% events
\newcommand{\E}[2][]{%
\protect\path{#2[#1]}%
}
% variable / fieldname ect
\newcommand{\V}[1]{%
\protect\path{#1}%
}
% interface
\newcommand{\I}[1]{{\ttfamily/#1/}}

% shell command
\newcommand{\cmd}[1]{{\\\ttfamily\$\xspace#1\xspace}}

\DeclareUrlCommand\path{\urlstyle{tt}}


%database table
\newcommand{\dbtable}[1]{{\ttfamily #1}\xspace}
\newcommand{\dbfield}[1]{{\ttfamily #1}\xspace}





%%% STANDART LAENGENANGABEN %%%%%%%%%%%%%%%%%%%%%%%%%%%%%%%%%%%%%%%%%%%%%%%%%%%
\newlength{\figwidth}
\setlength{\figwidth}{0.8\textwidth}

\newcommand{\todo}[1]{\textbf{\color{red}#1}\\}

